

\section{Conclusion}
Given a diverse set of motivations for variability studies, and a seemingly
intractable problem space when classical performance models are applied to
scientific workflows at extreme scale, we argue that a new performance model is
required. SOSflow was developed to enable the exploration and validation of new
performance models, especially those built for reasoning over high-level and
human-understandable semantic annotation that is affixed to all captured data and
events.
\subsection{Related Work}
\subsection{Recommendations}
Our research initiative is oriented towards on line and in
situ monitoring and  analytics. A system should be designed that is efficient
enough that it need not
be disabled for full-scale production runs of scientific workflows. This creates
the important case that the lessons learned through the use of such a
monitoring system are in fact applicable to future full-scale production runs,
and do not only apply to smaller sample test runs where the workflow's behavior
has been heavily perturbed by invasive monitoring technologies.
The conclusion goes here. this is more of the conclusion
\subsection{Future Work}

