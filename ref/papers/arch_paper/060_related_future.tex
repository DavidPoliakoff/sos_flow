\todofilebegin{060\_related\_future.tex}
%%%%%%%%%%%%%%%%%%%%%%%%%%%%%%%%%%%%%%%%%%%%%%%%%%%%%%%%%%%%%%%%%%%%%%%%%%%%%%
%%%%%%%%%%%%%%%%%%%%%%%%%%%%%%%%%%%%%%%%%%%%%%%%%%%%%%%%%%%%%%%%%%%%%%%%%%%%%%
%%%%%%%%%%%%%%%%%%%%%%%%%%%%%%%%%%%%%%%%%%%%%%%%%%%%%%%%%%%%%%%%%%%%%%%%%%%%%%

\section{Conclusion}
Given a diverse set of motivations for variability studies, and a
seemingly intractable problem space when classical performance models
are applied to scientific workflows at extreme scale, we argue that a
new performance model is required. SOSflow was developed to enable the
exploration and validation of new performance models, especially those
built for reasoning over high-level and human-understandable semantic
annotation that is affixed to all captured data and events.

%-----------------------------------------------------------------------------

\subsection{Related Work}



%-----------------------------------------------------------------------------

\subsection{Recommendations}
Our research initiative is oriented towards on line and in situ
monitoring and analytics. A system should be designed that is
efficient enough that it need not be disabled for full-scale
production runs of scientific workflows. This creates the important
case that the lessons learned through the use of such a monitoring
system are in fact applicable to future full-scale production runs,
and do not only apply to smaller sample test runs where the workflow's
behavior has been heavily perturbed by invasive monitoring
technologies.  \todo[inline]{FIX THIS SECTION!}  The conclusion goes
here. this is more of the conclusion.

%-----------------------------------------------------------------------------

\subsection{Future Work}

\textbf{automation}: Create an interactive script for integrating
SOSflow into existing Torque and Slurm job files to lower the barrier
to entry for new users and help check for the sanity of configurations
prior to wasted allocation time when mistakes are present.

\textbf{optimization}: Optimize the SOSflow codes for memory use and
data latency. Add mechanisms for throttling of data flow to increase
reliability in resource-constrained cases. Map out some best-fit
metrics for dedicating in situ resources to a monitoring platform for
some of the major extant compute clusters, and build this intelligence
into the SOSflow platform.

\textbf{TAUflow}: Continue extending the applicability of SOS and
the utility of SOSflow by integrating more of its capabilities directly
into existing HPC performance tools, especially regarding the online
analytics framework.

\textbf{soapy}: The Scalable Observation Analytics (Python) is a 
collection of scripts utilizing numpy and matplotlib to query the
SOSflow databases and produce visualizations. Some initial development
work here was utilized to produce the latency figures in this
paper.

\textbf{binning}: Add an sosd database feature to automatically
track ``bins'' of values based a user-selected granularity
parameter. A bin contains the earliest and latest rowid for each guid,
as well as a count of the number of updates to that guid, during the
timespan of that bin. Bins facilitate rapid queries of ``the overall
state of the system at time X'' and low-overhead time series plots
when rendering performance visualizations.

\textbf{data generalization}: Decouple the SOSflow components further
and provide alternative data flow and database storage strategies such
as EVPATH or Glasgow cache.

\textbf{synthetic workflow generator}: Continue work on the existing
code contribution for workflow generation. The current framework
already has a great deal of utility: An artibrarily complex directed
graph of applications are generated and linked together with a
continuous data flow using the ADIOS+FlexPath library. A python script
exists to generate the ADIOS configuration files automatically. It is
intended that the synthetic workflow tool can mature into a
general-purpose benchmarking and validation suide for exploring new
scientific workflow performance models under the MONA project.

\todofileend{060\_related\_future.tex}
