
%% bare_conf.tex
%% V1.3
%% 2007/01/11
%% by Michael Shell
%% See:
%% http://www.michaelshell.org/
%% for current contact information.
%%
%% This is a skeleton file demonstrating the use of IEEEtran.cls
%% (requires IEEEtran.cls version 1.7 or later) with an IEEE conference paper.
%%
%% Support sites:
%% http://www.michaelshell.org/tex/ieeetran/
%% http://www.ctan.org/tex-archive/macros/latex/contrib/IEEEtran/
%% and
%% http://www.ieee.org/

%%*************************************************************************
%% Legal Notice:
%% This code is offered as-is without any warranty either expressed or
%% implied; without even the implied warranty of MERCHANTABILITY or
%% FITNESS FOR A PARTICULAR PURPOSE! 
%% User assumes all risk.
%% In no event shall IEEE or any contributor to this code be liable for
%% any damages or losses, including, but not limited to, incidental,
%% consequential, or any other damages, resulting from the use or misuse
%% of any information contained here.
%%
%% All comments are the opinions of their respective authors and are not
%% necessarily endorsed by the IEEE.
%%
%% This work is distributed under the LaTeX Project Public License (LPPL)
%% ( http://www.latex-project.org/ ) version 1.3, and may be freely used,
%% distributed and modified. A copy of the LPPL, version 1.3, is included
%% in the base LaTeX documentation of all distributions of LaTeX released
%% 2003/12/01 or later.
%% Retain all contribution notices and credits.
%% ** Modified files should be clearly indicated as such, including  **
%% ** renaming them and changing author support contact information. **
%%
%% File list of work: IEEEtran.cls, IEEEtran_HOWTO.pdf, bare_adv.tex,
%%                    bare_conf.tex, bare_jrnl.tex, bare_jrnl_compsoc.tex
%%*************************************************************************

% *** Authors should verify (and, if needed, correct) their LaTeX system  ***
% *** with the testflow diagnostic prior to trusting their LaTeX platform ***
% *** with production work. IEEE's font choices can trigger bugs that do  ***
% *** not appear when using other class files.                            ***
% The testflow support page is at:
% http://www.michaelshell.org/tex/testflow/



% Note that the a4paper option is mainly intended so that authors in
% countries using A4 can easily print to A4 and see how their papers will
% look in print - the typesetting of the document will not typically be
% affected with changes in paper size (but the bottom and side margins will).
% Use the testflow package mentioned above to verify correct handling of
% both paper sizes by the user's LaTeX system.
%
% Also note that the "draftcls" or "draftclsnofoot", not "draft", option
% should be used if it is desired that the figures are to be displayed in
% draft mode.
%
\documentclass[10pt, conference, compsocconf]{IEEEtran}
% Add the compsocconf option for Computer Society conferences.
%
% If IEEEtran.cls has not been installed into the LaTeX system files,
% manually specify the path to it like:
% \documentclass[conference]{../sty/IEEEtran}





% Some very useful LaTeX packages include:
% (uncomment the ones you want to load)


% *** MISC UTILITY PACKAGES ***
%
%\usepackage{ifpdf}
% Heiko Oberdiek's ifpdf.sty is very useful if you need conditional
% compilation based on whether the output is pdf or dvi.
% usage:
% \ifpdf
%   % pdf code
% \else
%   % dvi code
% \fi
% The latest version of ifpdf.sty can be obtained from:
% http://www.ctan.org/tex-archive/macros/latex/contrib/oberdiek/
% Also, note that IEEEtran.cls V1.7 and later provides a builtin
% \ifCLASSINFOpdf conditional that works the same way.
% When switching from latex to pdflatex and vice-versa, the compiler may
% have to be run twice to clear warning/error messages.






% *** CITATION PACKAGES ***
%
%\usepackage{cite}
% cite.sty was written by Donald Arseneau
% V1.6 and later of IEEEtran pre-defines the format of the cite.sty package
% \cite{} output to follow that of IEEE. Loading the cite package will
% result in citation numbers being automatically sorted and properly
% "compressed/ranged". e.g., [1], [9], [2], [7], [5], [6] without using
% cite.sty will become [1], [2], [5]--[7], [9] using cite.sty. cite.sty's
% \cite will automatically add leading space, if needed. Use cite.sty's
% noadjust option (cite.sty V3.8 and later) if you want to turn this off.
% cite.sty is already installed on most LaTeX systems. Be sure and use
% version 4.0 (2003-05-27) and later if using hyperref.sty. cite.sty does
% not currently provide for hyperlinked citations.
% The latest version can be obtained at:
% http://www.ctan.org/tex-archive/macros/latex/contrib/cite/
% The documentation is contained in the cite.sty file itself.






% *** GRAPHICS RELATED PACKAGES ***
%
\ifCLASSINFOpdf
  % \usepackage[pdftex]{graphicx}
  % declare the path(s) where your graphic files are
  % \graphicspath{{../pdf/}{../jpeg/}}
  % and their extensions so you won't have to specify these with
  % every instance of \includegraphics
  % \DeclareGraphicsExtensions{.pdf,.jpeg,.png}
\else
  % or other class option (dvipsone, dvipdf, if not using dvips). graphicx
  % will default to the driver specified in the system graphics.cfg if no
  % driver is specified.
  % \usepackage[dvips]{graphicx}
  % declare the path(s) where your graphic files are
  % \graphicspath{{../eps/}}
  % and their extensions so you won't have to specify these with
  % every instance of \includegraphics
  % \DeclareGraphicsExtensions{.eps}
\fi
% graphicx was written by David Carlisle and Sebastian Rahtz. It is
% required if you want graphics, photos, etc. graphicx.sty is already
% installed on most LaTeX systems. The latest version and documentation can
% be obtained at: 
% http://www.ctan.org/tex-archive/macros/latex/required/graphics/
% Another good source of documentation is "Using Imported Graphics in
% LaTeX2e" by Keith Reckdahl which can be found as epslatex.ps or
% epslatex.pdf at: http://www.ctan.org/tex-archive/info/
%
% latex, and pdflatex in dvi mode, support graphics in encapsulated
% postscript (.eps) format. pdflatex in pdf mode supports graphics
% in .pdf, .jpeg, .png and .mps (metapost) formats. Users should ensure
% that all non-photo figures use a vector format (.eps, .pdf, .mps) and
% not a bitmapped formats (.jpeg, .png). IEEE frowns on bitmapped formats
% which can result in "jaggedy"/blurry rendering of lines and letters as
% well as large increases in file sizes.
%
% You can find documentation about the pdfTeX application at:
% http://www.tug.org/applications/pdftex





% *** MATH PACKAGES ***
%
%\usepackage[cmex10]{amsmath}
% A popular package from the American Mathematical Society that provides
% many useful and powerful commands for dealing with mathematics. If using
% it, be sure to load this package with the cmex10 option to ensure that
% only type 1 fonts will utilized at all point sizes. Without this option,
% it is possible that some math symbols, particularly those within
% footnotes, will be rendered in bitmap form which will result in a
% document that can not be IEEE Xplore compliant!
%
% Also, note that the amsmath package sets \interdisplaylinepenalty to 10000
% thus preventing page breaks from occurring within multiline equations. Use:
%\interdisplaylinepenalty=2500
% after loading amsmath to restore such page breaks as IEEEtran.cls normally
% does. amsmath.sty is already installed on most LaTeX systems. The latest
% version and documentation can be obtained at:
% http://www.ctan.org/tex-archive/macros/latex/required/amslatex/math/





% *** SPECIALIZED LIST PACKAGES ***
%
%\usepackage{algorithmic}
% algorithmic.sty was written by Peter Williams and Rogerio Brito.
% This package provides an algorithmic environment fo describing algorithms.
% You can use the algorithmic environment in-text or within a figure
% environment to provide for a floating algorithm. Do NOT use the algorithm
% floating environment provided by algorithm.sty (by the same authors) or
% algorithm2e.sty (by Christophe Fiorio) as IEEE does not use dedicated
% algorithm float types and packages that provide these will not provide
% correct IEEE style captions. The latest version and documentation of
% algorithmic.sty can be obtained at:
% http://www.ctan.org/tex-archive/macros/latex/contrib/algorithms/
% There is also a support site at:
% http://algorithms.berlios.de/index.html
% Also of interest may be the (relatively newer and more customizable)
% algorithmicx.sty package by Szasz Janos:
% http://www.ctan.org/tex-archive/macros/latex/contrib/algorithmicx/




% *** ALIGNMENT PACKAGES ***
%
%\usepackage{array}
% Frank Mittelbach's and David Carlisle's array.sty patches and improves
% the standard LaTeX2e array and tabular environments to provide better
% appearance and additional user controls. As the default LaTeX2e table
% generation code is lacking to the point of almost being broken with
% respect to the quality of the end results, all users are strongly
% advised to use an enhanced (at the very least that provided by array.sty)
% set of table tools. array.sty is already installed on most systems. The
% latest version and documentation can be obtained at:
% http://www.ctan.org/tex-archive/macros/latex/required/tools/


%\usepackage{mdwmath}
%\usepackage{mdwtab}
% Also highly recommended is Mark Wooding's extremely powerful MDW tools,
% especially mdwmath.sty and mdwtab.sty which are used to format equations
% and tables, respectively. The MDWtools set is already installed on most
% LaTeX systems. The lastest version and documentation is available at:
% http://www.ctan.org/tex-archive/macros/latex/contrib/mdwtools/


% IEEEtran contains the IEEEeqnarray family of commands that can be used to
% generate multiline equations as well as matrices, tables, etc., of high
% quality.


%\usepackage{eqparbox}
% Also of notable interest is Scott Pakin's eqparbox package for creating
% (automatically sized) equal width boxes - aka "natural width parboxes".
% Available at:
% http://www.ctan.org/tex-archive/macros/latex/contrib/eqparbox/





% *** SUBFIGURE PACKAGES ***
%\usepackage[tight,footnotesize]{subfigure}
% subfigure.sty was written by Steven Douglas Cochran. This package makes it
% easy to put subfigures in your figures. e.g., "Figure 1a and 1b". For IEEE
% work, it is a good idea to load it with the tight package option to reduce
% the amount of white space around the subfigures. subfigure.sty is already
% installed on most LaTeX systems. The latest version and documentation can
% be obtained at:
% http://www.ctan.org/tex-archive/obsolete/macros/latex/contrib/subfigure/
% subfigure.sty has been superceeded by subfig.sty.



%\usepackage[caption=false]{caption}
%\usepackage[font=footnotesize]{subfig}
% subfig.sty, also written by Steven Douglas Cochran, is the modern
% replacement for subfigure.sty. However, subfig.sty requires and
% automatically loads Axel Sommerfeldt's caption.sty which will override
% IEEEtran.cls handling of captions and this will result in nonIEEE style
% figure/table captions. To prevent this problem, be sure and preload
% caption.sty with its "caption=false" package option. This is will preserve
% IEEEtran.cls handing of captions. Version 1.3 (2005/06/28) and later 
% (recommended due to many improvements over 1.2) of subfig.sty supports
% the caption=false option directly:
%\usepackage[caption=false,font=footnotesize]{subfig}
%
% The latest version and documentation can be obtained at:
% http://www.ctan.org/tex-archive/macros/latex/contrib/subfig/
% The latest version and documentation of caption.sty can be obtained at:
% http://www.ctan.org/tex-archive/macros/latex/contrib/caption/




% *** FLOAT PACKAGES ***
%
%\usepackage{fixltx2e}
% fixltx2e, the successor to the earlier fix2col.sty, was written by
% Frank Mittelbach and David Carlisle. This package corrects a few problems
% in the LaTeX2e kernel, the most notable of which is that in current
% LaTeX2e releases, the ordering of single and double column floats is not
% guaranteed to be preserved. Thus, an unpatched LaTeX2e can allow a
% single column figure to be placed prior to an earlier double column
% figure. The latest version and documentation can be found at:
% http://www.ctan.org/tex-archive/macros/latex/base/



%\usepackage{stfloats}
% stfloats.sty was written by Sigitas Tolusis. This package gives LaTeX2e
% the ability to do double column floats at the bottom of the page as well
% as the top. (e.g., "\begin{figure*}[!b]" is not normally possible in
% LaTeX2e). It also provides a command:
%\fnbelowfloat
% to enable the placement of footnotes below bottom floats (the standard
% LaTeX2e kernel puts them above bottom floats). This is an invasive package
% which rewrites many portions of the LaTeX2e float routines. It may not work
% with other packages that modify the LaTeX2e float routines. The latest
% version and documentation can be obtained at:
% http://www.ctan.org/tex-archive/macros/latex/contrib/sttools/
% Documentation is contained in the stfloats.sty comments as well as in the
% presfull.pdf file. Do not use the stfloats baselinefloat ability as IEEE
% does not allow \baselineskip to stretch. Authors submitting work to the
% IEEE should note that IEEE rarely uses double column equations and
% that authors should try to avoid such use. Do not be tempted to use the
% cuted.sty or midfloat.sty packages (also by Sigitas Tolusis) as IEEE does
% not format its papers in such ways.





% *** PDF, URL AND HYPERLINK PACKAGES ***
%
%\usepackage{url}
% url.sty was written by Donald Arseneau. It provides better support for
% handling and breaking URLs. url.sty is already installed on most LaTeX
% systems. The latest version can be obtained at:
% http://www.ctan.org/tex-archive/macros/latex/contrib/misc/
% Read the url.sty source comments for usage information. Basically,
% \url{my_url_here}.





% *** Do not adjust lengths that control margins, column widths, etc. ***
% *** Do not use packages that alter fonts (such as pslatex).         ***
% There should be no need to do such things with IEEEtran.cls V1.6 and later.
% (Unless specifically asked to do so by the journal or conference you plan
% to submit to, of course. )


% correct bad hyphenation here
% \hyphenation{op-tical net-works semi-conduc-tor}


\begin{document}
%
% paper title
% can use linebreaks \\ within to get better formatting as desired
\title{SOSflow and a Semantic Workflow Performance Model\\
    for Understanding Variability in Scientific Workflows at Scale}

% \title{SOSflow : Observation, Introspection, Feedback, \\
%    and Control for Scientific Workflows}


% author names and affiliations
% use a multiple column layout for up to two different
% affiliations

\author{\IEEEauthorblockN{Chad Wood}
	\IEEEauthorblockN{Kevin Huck}
	\IEEEauthorblockN{Allen Malony}
\IEEEauthorblockA{Department of Computer and Information Science\\
University of Oregon\\
Eugene, OR United States\\
Email: cdw@cs.uoregon.edu}
% \and
% \IEEEauthorblockN{Kevin Huck}
% \IEEEauthorblockA{Department of Computer and Information Science\\
% line 2: University of Oregon\\
% line 3: Eugene, OR USA\\
% line 4: Email: khuck@cs.uoregon.edu}
}

% conference papers do not typically use \thanks and this command
% is locked out in conference mode. If really needed, such as for
% the acknowledgment of grants, issue a \IEEEoverridecommandlockouts
% after \documentclass

% for over three affiliations, or if they all won't fit within the width
% of the page, use this alternative format:
% 
%\author{\IEEEauthorblockN{Michael Shell\IEEEauthorrefmark{1},
%Homer Simpson\IEEEauthorrefmark{2},
%James Kirk\IEEEauthorrefmark{3}, 
%Montgomery Scott\IEEEauthorrefmark{3} and
%Eldon Tyrell\IEEEauthorrefmark{4}}
%\IEEEauthorblockA{\IEEEauthorrefmark{1}School of Electrical and Computer Engineering\\
%Georgia Institute of Technology,
%Atlanta, Georgia 30332--0250\\ Email: see http://www.michaelshell.org/contact.html}
%\IEEEauthorblockA{\IEEEauthorrefmark{2}Twentieth Century Fox, Springfield, USA\\
%Email: homer@thesimpsons.com}
%\IEEEauthorblockA{\IEEEauthorrefmark{3}Starfleet Academy, San Francisco, California 96678-2391\\
%Telephone: (800) 555--1212, Fax: (888) 555--1212}
%\IEEEauthorblockA{\IEEEauthorrefmark{4}Tyrell Inc., 123 Replicant Street, Los Angeles, California 90210--4321}}




% use for special paper notices
%\IEEEspecialpapernotice{(Invited Paper)}




% make the title area
\maketitle


\begin{abstract}
SOSflow provides a run-time system and performance model designed to enable the 
characterization and analysis of complex scientific workflow performance at scale.


\end{abstract}

\begin{IEEEkeywords}
hpc; exascale; in situ; performance; monitoring; introspection; scientific 
workflow;

\end{IEEEkeywords}


% For peer review papers, you can put extra information on the cover
% page as needed:
% \ifCLASSOPTIONpeerreview
% \begin{center} \bfseries EDICS Category: 3-BBND \end{center}
% \fi
%
% For peerreview papers, this IEEEtran command inserts a page break and
% creates the second title. It will be ignored for other modes.
\IEEEpeerreviewmaketitle


\section{Introduction}
% no \IEEEPARstart
Here we set up the problem space we are addressing.\\
\\
It will be a good time!\\
\\
\textit{[NOTE: Come back and flesh out the INTRO and ABSTRACT at the end, once 
we're sure the content we are after is a part of the paper.]}


\section{Motivations for Tracking Variability}
\textit{[NOTE: More than simply demonstrating that there are different and 
possibly incompatible motivations for variability research, this section is 
setting up the necessity for a new performance model.]}\\
Due to the novelty of this research area, there will be opportunities to 
further refine the way in which research is taxonomized, and to establish 
a set of definitions that provide adequate coverage of the conceptually distinct 
types of variability studies. Though such preliminaries are not our primary 
purpose here, it is a  useful  exercise to consider some of the different 
purposes behind current studies of performance variability. The design choices 
made for the SOSflow system and its inherent workflow performance model emerged 
from reflecting on the general nature of variability studies, the 
divergent purposes that create tensions between the efforts being 
undertaken in this new field, and the anticipated realities of writing 
infrastructure codes for exascale HPC clusters.
\subsection{Why Variability is Tracked}
Many factors have contributed to the emergence of variability studies as an 
important research topic within the HPC community. At a low-level, HPC node 
engineering has grown in complexity and sophistication, many on-core processor 
behaviors that used to be isolated, synchronous, and predictable, are now 
interrelated, data-driven, asynchronous, and impossible to predict a priori. As 
core density increases on the nodes of a cluster (TODO: Cite paper showing this 
is the necessary direction of exascale, HANK has it) the unpredictability and inconsistency 
of the hardware itself becomes an increasingly significant contributor to 
observed variability. Hardware is an important source to consider when it comes 
to variability because there is almost nothing that can be done to control for 
it, the noise has to be admitted in the results, and therefore is an important 
part of the output of any performance-related experiment.\\
\\
Using higher-levels of  description, variability in workflow 
performance can be introduced by many different sources:
\begin{itemize}
    \item Versions of software libraries across can differ across clusters, or 
    even the same cluster across time.
    \item Tuning factors to extract maximum 
    performance from a code can vary across clusters even if the code and the 
    data do not change. Shared filesystem performance, data transport methods, 
    available per-process memory, 
    co-processor presence and architecture, ...and more, all can be responsible 
    for influencing sensitivity to novel tuning factors.
    \item Concurrent activity elsewhere on the same cluster, activity that will 
    necessarily vary between every iteration of the workflow, may be having a 
    significant impact on observed performance.
    \item At extreme scales, parts of a simulation are almost guaranteed to 
    fail due to the marginal failure rates of hardware components approaching 
    absolute certainty as the number of involved components increases. These 
    failures cannot 
    be accurately predicted a priori, and the design constraints that account 
    for and respond to them introduce performance perturbation and further 
    complexity.
    \item \ldots \textit{[NOTE]}
    \item \ldots \textit{[NOTE]}
\end{itemize}
\subsubsection{Awareness of Fine-Grained Performance Jitter}
\textit{[NOTE: This section is purposed with showing that even episodic 
close analysis of individual components at small scale will have issues with 
significant variability. There are tractible 
concerns (demonstrating sensitivity to var.) and inherently intractible 
concerns (controlling var. across runs).]}\\
Even in cases where nothing can be done to influence the variations in 
performance, it is important to recognize that performance variability is 
present and to  attempt to both quantify it and render reasonable attributions 
of its source[s]. Significant variability between runs is seen (SCHULTZ 
paper/graph) when tracking a single simple experiment, even if the input data, 
job queue  parameters, and hardware allocation is held constant. At extreme 
scales, on line detection and attribution of variability of a 
scientific workflow will require well-annotated metadata to facilitate "apples 
to apples" comparisons driven by unsupervised machine learning rather than a 
priori developer knowledge or offline centralized analysis. It is important to 
characterize a code's sensitivity to variability, as well as a cluster's 
propensity for creating performance variability.
\subsubsection{Accuracy in Performance Research}
\textit{[NOTE: This section is intended to motivate a new model for 
performance when 
considering scientific workflows on exascale systems. Nail the coffin shut on 
existing performance research validation tools and techniques. They are 
intractible, per the above notes, and irrelevant, per this discussion]}\\
Performance research is principally concerned with decreasing the resource 
consumption and compute time required by low-level components and libraries 
that are used when constructing higher-level scientific workflows. For example: 
In order to 
validate a 5 percent increase in some code's performance, it will be necessary 
to show that  performance increase was observed across a vast array of runs and 
hardware allocations, especially if it is the case that a particular HPC 
cluster (when in an overall state similar to the one it was in during those 
workflow runs) has a history of performance variability with any statistical 
significance relative to the observed performance gain. When it comes to the 
behavior of codes at extreme scales, accuracy validation using traditional 
models of component-based performance analysis will become cost prohibitive in 
both allocation consumption and developer time.
\subsubsection{Reproduction of Experimental Results}
Scientific workflows attempt to yield results that have truth-coorespondence 
with the physical world with some overt degree of significance.

\subsection{How Variability is Tracked}
\subsubsection{Concepts and Methods}
\subsubsection{Existing Models and Tools}

\subsection{The Utility a Comprehensive Workflow Performance Model}
\subsubsection{Attribution}
\subsubsection{Resource Requirement Prediction}
\subsubsection{Automated Component Performance Tuning}
Don't want conflicting optimizer purposes. Need to know where the hotspots 
*really* are and not 
\subsubsection{Intelligent Compiler Hints}
Don't have to burn 1,000,000 hours of allocation to learn that you just re-created last year's mistaken 

\subsubsection{Intelligent Job Scheduling}



\section{A Performance Model for Scientific Workflows}
Traditionally, HPC performance monitoring is focused on low-level efficiency of 
an application binary on some particular iron.  Higher-level systems (TACC 
Stats) allow the tracking and exploration of execution wall-time for various 
library versions, integration of multiple modalities of information (LDMS) like 
program invocation or work allocation across a cluster as informed by network 
congestion statistics, and other hybridized or meta-execution data points. 
Low-level metrics are more naturally suited for off-line episodic 
performance analysis of individual workflow components, but cannot yield insight 
into the run-time performance of a complex workflow.  Characterizing and 
understanding the emergent properties of a workflow comprised 
of many components that are interacting asynchronously across a distributed HPC 
cluster requires taking a new approach to the problem, especially when 
considering the extreme scales of parallelism to which scientific workflows are 
being driven. 
\subsection{Workflow Variability is Revealed by Invariant Meaning}

\subsection{Levels of Description and "The View from Anywhere"}
Not knowing \textit{a priori} what component or layer of the workflow will be 
responsible for the introduction of variability, the workflow performance model 
needs to be populated by a diversity of information sources that provide 
metrics with tailored metadata and both logical and concrete events, arriving 
in real-time from many different layers of activity in a workflow:
\begin{itemize}
    \item Simulation
    \item Algorithm
    \item Application
    \item Libraries
    \item Environment
    \item Developer Tools
    \item Operating System
    \item Node Hardware
    \item Network
    \item Enclave
    \item Cluster
    \item Workflow
    \item Epoch	
\end{itemize}
Each of these layers constitutes a \textit{level of description} for workflow 
performance. If the Simulation layer produced data representing the evolution 
of a system over N seconds of simulated time, and the Application layer 
requires M seconds of real-world computation to yield that data, the 
relationship between N and M should be a valid performance metric to report, 
compare across runs, and to attempt to generally optimize through parameter 
convergence.
\subsection{Semantics}
All information that is gathered by the monitoring system should be annotated 
as richly as possible to maximize its usefulness when performing analytics. 
Hand-annotated codes will have the most to offer an analytics engine,  
values that are tracked will be able to carry a full spectrum of high-level 
tags that express what that data point means and what could be expected of it, 
in structure preserving a human programmer or user's understanding while also
being compatible with unsupervised machine-learning tests for significance and 
other advanced analysis techniques.\\
\\
Any episodic performance measurement, such as run-time TAU instrumentation, can 
also be injected into the SOSflow engine, and the SOSflow runtime will be able to 
differentiate from information pushed directly by a layer, and information that 
is being captured by middle-ware tools.\\
\\
\textit{[NOTE: Insert table heres of some of the enum values alongside 
plain-english descriptions.]}\\
\\
Semantic information is local to the "publication handle" (pub) 
created by a source that is contributing to SOSflow.  Sources can create 
multiple pubs to distinctly represent potential compound or complex roles. Pubs 
carry their own pub-wide semantic markups, including 
the origin layer, a role within that layer, and information 
about the node that the source process is running on. Semantic markups are then 
nested inside of the pub handle, as each value that is pushed 
into the SOSflow system through a pub handle also comes with a rich set of 
high-level semantic tags that stay affiliated with it over time. While deep 
off-line data analytics 
can reveal unforseen correlations between various aspects of the workflow or the 
data set it is operating over, the interest in real-time analytics and 
performance tuning gives value to expressing "relationship hints" between 
values tracked by the system. These hints can be used to direct in situ 
analytics and identify deviations from expectations; anything that can be 
overtly identified as an expectation for a value can be used to narrow the 
search space when doing unsupervised machine learning over gathered workflow 
performance data.

\section{SOSflow}
Applying a novel semantic workflow performance model to actual run-time 
environments necessitated the development of new infrastructure software, and 
so the second contribution to the general challenges of monitoring scientific 
workflows is the SOSflow software artifact.\\
\\
The initial design goals of SOSflow are:
\begin{enumerate}
\item Facilitate capture of scientific workflow performance data and events in 
situ (i.e. "on node") from a variety of sources and perspectives at the same 
time.
\item Annotate the gathered data's semantic meaning with expressive 
"high-level" 
tags that facilitate contextualization and introspective analytics.
\item Store the captured data on node in a way that can be searched with 
dynamic queries in real-time as well as being suitable for long-term 
centralized 
archival.
\end{enumerate}
\ldots the SQLite3 open-source DB engine has been selected 
during the initial development phase for on-node storage, but the API is 
modularized and a different 
format/mechanism could be used instead.  On-node storage is logically separated 
off-node data transport mechanisms, but they could be the same backend if the 
enabling technology supports that.\\
\\
SOSflow is divided into two main parts in it's current incarnation:
\begin{itemize}
\item sosd - Daemon process running on each node
\item libsos - Library of routines for interacting with the daemon
\end{itemize}
The sosd daemon launches before a scientific workflow begins, and passively 
listens on 
a socket.  The port that is available to all of the sources on any given node 
is found in the "SOS\_CMD\_PORT" environment 
variable.  Many programs and layers of programs can connect to the daemon and 
send information in. Library-to-daemon communication is invisible to the SOSflow 
user and happens entirely on-node using a simple stateless protocol.  When the 
workflow is completed, the sosd\_stop tool is executed on each node, it signals 
the daemon to flush buffers and close down.
\subsection{Behavior}
\textit{[NOTE: Here I can introduce a narrative or a figure or both, to explain 
how a typical SOSflow session should happen on the cluster. I am skipping this 
for now, to be able to release this document for revision and comment. This 
section is obvious and easy to both write and edit, but I want to get this out 
to you guys.]}
\subsection{Implementation}
\subsubsection{Language}C
\subsubsection{External Requirements}
\begin{itemize}
    \item cmake
    \item pthreads
    \item MPI
    \item Sqlite3 (*May change across implementations)
\end{itemize}
\subsubsection{Key Source Files}
See Table 1.

\begin{table}[!t]
%% increase table row spacing, adjust to taste
\renewcommand{\arraystretch}{1.3}
\caption{Key Source Files for SOSflow}
\label{table_example}
\centering
\begin{tabular}{|c|c|}
\hline
sos.h / sos.c & Becomes libsos, the core functions of SOSflow\\
\hline
sosd.h / sosd.c & SOSflow daemon\\
\hline
sos\_debug.h & Debugging off / on (level) knobs\\
\hline
demo\_app.c & The "Hello, world." of SOSflow value injection\\
\hline
sosd\_cloud\_mpi.c & Off-node transport using simple MPI calls\\
\hline
sosd\_db\_sqlite.c & On-node DB creation and value-injection\\
\hline
\end{tabular}
\end{table}

\subsection{Limitations and Concerns}
The value of SOSflow is directly proportional to the quantity and quality of the 
sources that are pushing semantically-annotated data into it. At this time the 
use-case development has centered on the ADIOS scalable I/O library, due to its 
deep integration with many existing scientific workflows and industry standard 
tools  like VisIt. ADIOS instrumentation is a source of significant
observable data and an actionable target for effective feedback and control.\\
\\
The more sources that are instrumented with SOSflow the less limited the workflow 
performance model will be. When only one or two layers are instrumented, the 
benefits of the semantic annotation and the workflow performance model are 
naturally limited.\\
\\
The SOSflow software artifact is in its initial development cycle and has room 
for improvement in various software engineering facets:  
memory subsystem interaction, diversity of storage and transport mechanisms, 
and quality-of-service guarantees to both producers and consumers of 
SOSflow data.
\subsection{Results of Initial Benchmarking}
\textit{[NOTE: We need to decide what we can put in here (or if we should even 
have this section at this time) ... it could be something as simple as 
PERCENTAGE of difference in the execution time of a synthetic workflow when the 
SOSflow system is activated compared to when it is not activated. I fully 
expect the perturbation to be EXTREMELY LOW, especially if values are being 
injected non-continuously and at natural pauses in computation like at the top 
of loops]}
\section{Conclusion}
Given a diverse set of motivations for variability studies, and a seemingly 
intractable problem space when classical performance models are applied to 
scientific workflows at extreme scale, we argue that a new performance model is 
required. SOSflow was developed to enable the exploration and validation of new 
performance models, especially those built for reasoning over high-level and 
human-understandable semantic annotation that is affixed to all captured data and 
events.
\subsection{Recommendations}
Our research initiative is oriented towards on line and in 
situ monitoring and  analytics. A system should be designed that is efficient 
enough that it need not 
be disabled for full-scale production runs of scientific workflows. This creates 
the important case that the lessons learned through the use of such a 
monitoring system are in fact applicable to future full-scale production runs, 
and do not only apply to smaller sample test runs where the workflow's behavior 
has been heavily perturbed by invasive monitoring technologies.
The conclusion goes here. this is more of the conclusion
\subsection{Future Work}

% use section* for acknowledgement
\section*{Acknowledgment}
Research has been conducted under the following grants:\\
\textit{TODO: INSERT GRANT INFORMATION HERE}\\
\\
The authors would like to thank:\\
NAME of Georgia Tech University\\
Hasan Abassi of Argonne National Lab\\
Todd Gamblin of Lawrence Livermore National Lab

\begin{thebibliography}{1}

\bibitem{IEEEhowto:kopka}
H.~Kopka and P.~W. Daly, \emph{A Guide to \LaTeX}, 3rd~ed.\hskip 1em plus
  0.5em minus 0.4em\relax Harlow, England: Addison-Wesley, 1999.

\end{thebibliography}

% that's all folks
\end{document}




% An example of a floating figure using the graphicx package.
% Note that \label must occur AFTER (or within) \caption.
% For figures, \caption should occur after the \includegraphics.
% Note that IEEEtran v1.7 and later has special internal code that
% is designed to preserve the operation of \label within \caption
% even when the captionsoff option is in effect. However, because
% of issues like this, it may be the safest practice to put all your
% \label just after \caption rather than within \caption{}.
%
% Reminder: the "draftcls" or "draftclsnofoot", not "draft", class
% option should be used if it is desired that the figures are to be
% displayed while in draft mode.
%
%\begin{figure}[!t]
%\centering
%\includegraphics[width=2.5in]{myfigure}
% where an .eps filename suffix will be assumed under latex, 
% and a .pdf suffix will be assumed for pdflatex; or what has been declared
% via \DeclareGraphicsExtensions.
%\caption{Simulation Results}
%\label{fig_sim}
%\end{figure}

% Note that IEEE typically puts floats only at the top, even when this
% results in a large percentage of a column being occupied by floats.


% An example of a double column floating figure using two subfigures.
% (The subfig.sty package must be loaded for this to work.)
% The subfigure \label commands are set within each subfloat command, the
% \label for the overall figure must come after \caption.
% \hfil must be used as a separator to get equal spacing.
% The subfigure.sty package works much the same way, except \subfigure is
% used instead of \subfloat.
%
%\begin{figure*}[!t]
%\centerline{\subfloat[Case I]\includegraphics[width=2.5in]{subfigcase1}%
%\label{fig_first_case}}
%\hfil
%\subfloat[Case II]{\includegraphics[width=2.5in]{subfigcase2}%
%\label{fig_second_case}}}
%\caption{Simulation results}
%\label{fig_sim}
%\end{figure*}
%
% Note that often IEEE papers with subfigures do not employ subfigure
% captions (using the optional argument to \subfloat), but instead will
% reference/describe all of them (a), (b), etc., within the main caption.


% An example of a floating table. Note that, for IEEE style tables, the 
% \caption command should come BEFORE the table. Table text will default to
% \footnotesize as IEEE normally uses this smaller font for tables.
% The \label must come after \caption as always.
%
%\begin{table}[!t]
%% increase table row spacing, adjust to taste
%\renewcommand{\arraystretch}{1.3}
% if using array.sty, it might be a good idea to tweak the value of
% \extrarowheight as needed to properly center the text within the cells
%\caption{An Example of a Table}
%\label{table_example}
%\centering
%% Some packages, such as MDW tools, offer better commands for making tables
%% than the plain LaTeX2e tabular which is used here.
%\begin{tabular}{|c||c|}
%\hline
%One & Two\\
%\hline
%Three & Four\\
%\hline
%\end{tabular}
%\end{table}


% Note that IEEE does not put floats in the very first column - or typically
% anywhere on the first page for that matter. Also, in-text middle ("here")
% positioning is not used. Most IEEE journals/conferences use top floats
% exclusively. Note that, LaTeX2e, unlike IEEE journals/conferences, places
% footnotes above bottom floats. This can be corrected via the \fnbelowfloat
% command of the stfloats package.



% conference papers do not normally have an appendix


% trigger a \newpage just before the given reference
% number - used to balance the columns on the last page
% adjust value as needed - may need to be readjusted if
% the document is modified later
%\IEEEtriggeratref{8}
% The "triggered" command can be changed if desired:
%\IEEEtriggercmd{\enlargethispage{-5in}}

% references section

% can use a bibliography generated by BibTeX as a .bbl file
% BibTeX documentation can be easily obtained at:
% http://www.ctan.org/tex-archive/biblio/bibtex/contrib/doc/
% The IEEEtran BibTeX style support page is at:
% http://www.michaelshell.org/tex/ieeetran/bibtex/
%\bibliographystyle{IEEEtran}
% argument is your BibTeX string definitions and bibliography database(s)
%\bibliography{IEEEabrv,../bib/paper}
%
% <OR> manually copy in the resultant .bbl file
% set second argument of \begin to the number of references
% (used to reserve space for the reference number labels box)
