%\todofilebegin{060\_related\_future.tex}
%%%%%%%%%%%%%%%%%%%%%%%%%%%%%%%%%%%%%%%%%%%%%%%%%%%%%%%%%%%%%%%%%%%%%%%%%%%%%%
%%%%%%%%%%%%%%%%%%%%%%%%%%%%%%%%%%%%%%%%%%%%%%%%%%%%%%%%%%%%%%%%%%%%%%%%%%%%%%
%%%%%%%%%%%%%%%%%%%%%%%%%%%%%%%%%%%%%%%%%%%%%%%%%%%%%%%%%%%%%%%%%%%%%%%%%%%%%%

\section{Conclusion}

%%%%%
Given a diverse set of motivations for variability studies, and a
seemingly intractable problem space when classical performance models
are applied to scientific workflows at extreme scale, we argue that a
new performance model is required.
%
SOSflow was developed to enable the exploration and validation of new
performance models, especially those built for reasoning over
high-level and human-understandable semantic annotation that is
affixed to all captured data and events.
%


%-----------------------------------------------------------------------------

\subsection{Future Work}

%%%%%
%%%%%

%\todofileend{060\_related\_future.tex}
