
%%%%%%%%%%%%%%%%%%%%%%%%%%%%%%%%%%%%%%%%%%%%%%%%%%%%%%%%%%%%%%%%%%%%%%%%%%%%%%%
%%%%%%%%%%%%%%%%%%%%%%%%%%%%%%%%%%%%%%%%%%%%%%%%%%%%%%%%%%%%%%%%%%%%%%%%%%%%%%%
%%%%%%%%%%%%%%%%%%%%%%%%%%%%%%%%%%%%%%%%%%%%%%%%%%%%%%%%%%%%%%%%%%%%%%%%%%%%%%%

%\section{Future Work} %-------------------------------------------------------%
%
%\textbf{Development}: Continue refining and expanding the core SOSflow
%libraries and the SOS workflow performance model.
\\
%\textbf{Optimization}: Optimize the SOSflow codes for memory use
%and data latency. Add mechanisms for throttling of data flow to
%increase reliability in resource-constrained cases. Map out 
%best-fit metrics for dedicating in situ resources to a monitoring
%platform for the major extant and proposed compute clusters, and build this
%intelligence into the SOSflow implementation.
\\
%\textbf{Integration}: Explore options for deployment and integration with
%existing HPC monitoring and analytics codes at LLNL and other
%national laboratories.
%
%

\section{Conclusion}
%
The SOS Model presented is online, scalable and supports a global information space. 
%
The model enables online in situ characterization and analysis of complex high-performance computing applications. 
%
The operational system SOSflow provides a flexible research platform for investigating the properties of existing and future scientific workflows, supporting both current and future scales of execution.
%
Experimental results demonstrated that SOSflow is capable of
observation, introspection, feedback and control of complex scientific
workflows, and that it has desirable scaling properties.
%

As part of future development, we aim to continue refining and expanding the core SOSflow libraries and the SOS workflow performance model.
%
The SOSflow codes can be optimized for memory use
and data latency. 
%
Mechanisms can be added for throttling of data flow to
increase reliability in resource-constrained cases. 
%
Futher, mapping out of best-fit metrics for dedicating in situ resources to a monitoring platform for the major extant and proposed compute clusters.
%
Additionally, we plan on exploring options for deployment and integration with
existing HPC monitoring and analytics codes at LLNL and other
national laboratories.


%


%%%%%%%%%%%%%%%%%%%%%%%%%%%%%%%%%%%%%%%%%%%%%%%%%%%%%%%%%%%%%%%%%%%%%%%%%%%%%%%
%%%%%%%%%%%%%%%%%%%%%%%%%%%%%%%%%%%%%%%%%%%%%%%%%%%%%%%%%%%%%%%%%%%%%%%%%%%%%%%
%%%%%%%%%%%%%%%%%%%%%%%%%%%%%%%%%%%%%%%%%%%%%%%%%%%%%%%%%%%%%%%%%%%%%%%%%%%%%%%
\section{Acknowledgments}

The research report was supported by a grant (DE-SC0012381) from the
Department of Energy, Scientific Data Management, Analytics, and
Visualization (SDMAV), for ``Performance Understanding and Analysis
for Exascale Data Management Workflows.''

Part of this work was performed under the auspices of the
U.S. Department of Energy by Lawrence Livermore National Laboratory
under Contract DE-AC52-07NA27344 (LLNL-CONF-XXXXXX).



%%%
%%%  EOF
%%%
