
%%%%%%%%%%%%%%%%%%%%%%%%%%%%%%%%%%%%%%%%%%%%%%%%%%%%%%%%%%%%%%%%%%%%%%%%%%%%%%%
%%%%%%%%%%%%%%%%%%%%%%%%%%%%%%%%%%%%%%%%%%%%%%%%%%%%%%%%%%%%%%%%%%%%%%%%%%%%%%%
%%%%%%%%%%%%%%%%%%%%%%%%%%%%%%%%%%%%%%%%%%%%%%%%%%%%%%%%%%%%%%%%%%%%%%%%%%%%%%%

\section{Future Work} %-------------------------------------------------------%
%
\textbf{SOS/SOSflow}: Continue refining and expanding the core SOSflow
libraries and the SOS workflow performance model.
\\
\textbf{Automation}: Create an interactive script for integrating
SOSflow into existing Torque and Slurm job files to lower the barrier
to entry for new users and help check for the sanity of configurations,
avoiding wasted allocation time if mistakes are present.
\\
\textbf{SOApy}: The Scalable Observation Analytics (Python) is a
collection of scripts utilizing numpy and matplotlib to query the
SOSflow databases and produce visualizations. Some initial development
work here was utilized to produce the latency figures in this paper.
\\
\textbf{Data Store Decoupling}: Decouple the
SOSflow components further and provide alternative data flow and
database storage strategies such as EVPATH, Glasgow Cache, or
Cassandra+Spark.
\\
\textbf{Synthetic Workflow Generator}: Continue effort on the 
synthetic workflow generation tools for workflow modeling.
\\
\textbf{Optimization}: Optimize the SOSflow codes for memory use
and data latency. Add mechanisms for throttling of data flow to
increase reliability in resource-constrained cases. Map out some
best-fit metrics for dedicating in situ resources to a monitoring
platform for some of the major extant compute clusters, and build this
intelligence into the SOSflow platform.
\\
\textbf{Integration}: Explore options for deployment and integration with
existing HPC monitoring and analytics codes at LLNL and other
national laboratories.
%
%

\section{Conclusion}
%
SOSflow provides a flexible research platform for investigating the
properties of existing and future scientific workflows, supporting
both current and future scales of execution.
%
Experimental results demonstrated that SOSflow is capable of
observation, introspection, feedback and control of complex scientific
workflows, and that it has desirable scaling properties.
%
%


%%%%%%%%%%%%%%%%%%%%%%%%%%%%%%%%%%%%%%%%%%%%%%%%%%%%%%%%%%%%%%%%%%%%%%%%%%%%%%%
%%%%%%%%%%%%%%%%%%%%%%%%%%%%%%%%%%%%%%%%%%%%%%%%%%%%%%%%%%%%%%%%%%%%%%%%%%%%%%%
%%%%%%%%%%%%%%%%%%%%%%%%%%%%%%%%%%%%%%%%%%%%%%%%%%%%%%%%%%%%%%%%%%%%%%%%%%%%%%%
\section{Acknowledgments}
Part of this work was performed under the auspices of the
U.S. Department of Energy by Lawrence Livermore National Laboratory
under Contract DE-AC52-07NA27344 (LLNL-CONF-XXXXXX).



%%%
%%%  EOF
%%%
