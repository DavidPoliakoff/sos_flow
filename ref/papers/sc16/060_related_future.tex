%\todofilebegin{060\_related\_future.tex}
%%%%%%%%%%%%%%%%%%%%%%%%%%%%%%%%%%%%%%%%%%%%%%%%%%%%%%%%%%%%%%%%%%%%%%%%%%%%%%
%%%%%%%%%%%%%%%%%%%%%%%%%%%%%%%%%%%%%%%%%%%%%%%%%%%%%%%%%%%%%%%%%%%%%%%%%%%%%%
%%%%%%%%%%%%%%%%%%%%%%%%%%%%%%%%%%%%%%%%%%%%%%%%%%%%%%%%%%%%%%%%%%%%%%%%%%%%%%

\section{Conclusion}

%%%%%
Given a diverse set of motivations for variability studies, and a
seemingly intractable problem space when classical performance models
are applied to scientific workflows at extreme scale, we argue that a
new performance model is required.
%
SOSflow was developed to enable the exploration and validation of new
performance models, especially those built for reasoning over
high-level and human-understandable semantic annotation that is
affixed to all captured data and events.
%
Preliminary results indicate that SOSflow is a stable, efficient,
fully-functioning system, capable of serving a variety of interests in
the HPC community.


%-----------------------------------------------------------------------------

\subsection{Related Work}
\todo[inline]{
ptrace
CM*
TAUg
Caliper
TACC Stats
LDMS
Cache
Ganglia
Nagios
}

\section{Related Work}

%%%%%
Traditionally, HPC research into enhancing performance has been
focused on low-level efficiency of an application or library on some
particular mechine, with tools like TAU bringing HPC developers ever
closer to optimal runs on specific machines.
%
Low-level metrics are naturally suited for off-line episodic
performance analysis of individual workflow components.
%
Such deep instrumentation is necessarily invasive and can dictate
rather than capture the observed performance of the instrumented
application when it is run at scale or required to engage in
significant amounts of interactivity.
%
Doing principle components analysis on unconditioned data is also
computationally expensive and unsuitible for a runtime environment.
%
While differing from them significantly, as discussed in the prior
sections, SOSflow draws some concepts from many existing tools.
%
These tools, and many other performance monitoring tools, are
well-implemented, tested, maintained, deployed and regularly used for
performance research studies.
%%%%%

%%%%%
Higher-level systems like TACC Stats \cite{evans2014comprehensive}
allow the tracking and exploration of execution wall-time for
applications compiled using various library versions.
%
The popular Lightweight Distributed Metric System (LDMS)
\cite{agelastos2014lightweight} provides basic integration of multiple
modalities of data in real-time, triggering program invocation or
shaping work allocation across a cluster as informed by network
congestion statistics, and other hybridized or meta-execution data
points.
%%%%%



%-----------------------------------------------------------------------------

\subsection{Future Work}

%%%%%
\textbf{SOS/SOSflow}: Continue refining and expanding the core SOSflow
libraries and the SOS workflow performance model.
%
\textbf{Automation}: Create an interactive script for integrating
SOSflow into existing Torque and Slurm job files to lower the barrier
to entry for new users and help check for the sanity of configurations
prior to wasted allocation time when mistakes are present.
%
\textbf{Optimization}: Optimize the SOSflow codes for memory use and
data latency. Add mechanisms for throttling of data flow to increase
reliability in resource-constrained cases.
%
Map out some best-fit metrics for dedicating in situ resources to a
monitoring platform for some of the major extant compute clusters, and
build this intelligence into the SOSflow platform.
%
\textbf{Integration}: Explore options for deployment and integration
with existing HPC monitoring and analytics codes at LLNL during the
summer of 2016.
%
\textbf{TAUflow}: Continue extending the applicability of SOS and
the utility of SOSflow by integrating more of its capabilities directly
into existing HPC performance tools, especially regarding the online
analytics framework.
%
\textbf{SOApy}: The Scalable Observation Analytics (Python) is a 
collection of scripts utilizing numpy and matplotlib to query the
SOSflow databases and produce visualizations.
%
Some initial development work here was utilized to produce the latency
figures in this paper.
%
\textbf{Binning}: Add an sosd database feature to automatically
track ``bins'' of values based a user-selected granularity
parameter.
%
A bin contains the earliest and latest rowid for each guid, as well as
a count of the number of updates to that guid, during the timespan of
that bin.
%
Bins facilitate rapid queries of ``the overall state of the system at
time X'' and low-overhead time series plots when rendering performance
visualizations.
%
\textbf{Data Store Decoupling}: Decouple the SOSflow components further
and provide alternative data flow and database storage strategies such
as EVPATH, Glasgow cache, or process-local JSON file dumps that can
later be synthesized into databases.
%
\textbf{Synthetic Workflow Generator}: Continue work on the existing
code contribution for workflow generation.
%
The current framework already has a great deal of utility: An
artibrarily complex directed graph of applications are generated and
linked together with a continuous data flow using the ADIOS+FlexPath
library.
%
A python script exists to generate the ADIOS configuration files
automatically.
%
It is intended that the synthetic workflow tool can mature into a
general-purpose benchmarking and validation suide for exploring new
scientific workflow performance models under the MONA project.
%%%%%

%\todofileend{060\_related\_future.tex}
