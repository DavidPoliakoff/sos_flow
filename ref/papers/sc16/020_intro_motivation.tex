%\todofilebegin{020\_intro\_motivation.tex}
%%%%%%%%%%%%%%%%%%%%%%%%%%%%%%%%%%%%%%%%%%%%%%%%%%%%%%%%%%%%%%%%%%%%%%%%%%%%%%
%%%%%%%%%%%%%%%%%%%%%%%%%%%%%%%%%%%%%%%%%%%%%%%%%%%%%%%%%%%%%%%%%%%%%%%%%%%%%%
%%%%%%%%%%%%%%%%%%%%%%%%%%%%%%%%%%%%%%%%%%%%%%%%%%%%%%%%%%%%%%%%%%%%%%%%%%%%%%
% NOTE: no \IEEEPARstart

%%%%%
\section{Introduction}
Modern clusters for parallel computing are complex environments and
the high-performance applications that run on them do so often with
little insight about their or the system's behavior.
%
This is not to say that information is unavailable.  After all,
sophisticated parallel measurement systems can capture performance and
power data for characterization, analysis, and tuning purposes, but
the infrastructure for observation of these systems is not intended
for general use.
%
Rather, it is specialized for certain types of performance information
and typically does not allow online processing.
%
Other information sources of interest might include the
operating system (OS), network hardware, runtime services, or the
parallel application itself.
%
Cluster monitoring systems like Ganglia \cite{massie2004ganglia} or
Nagios \cite{katsaros2011building} collect and process data about the
performance and health of cluster-wide resources, but do not provide
sufficient fidelity to capture the complex interplay between
applications competing for shared resources.
%
In contrast, the Lightweight Distributed Metric Service
\cite{agelastos2014lightweight} (LDMS) attempts to capture system data
continuously to obtain insight into behavioral characteristics of
individual applications with respect to their resource utilization.
%
However, neither of these provides a framework that can be configured
with and used directly by the application, nor allow for semantic
encoding of multiple observation sources.
%%%%%%%

%%%%%%%
Our general interest is in parallel application monitoring: the
observation, introspection, and possible adaptation of an application
during its execution.
%
Application monitoring has several requirements.  Because information
could come from different sources and be used for different purposes,
it is important to have a flexible means for information to be
provided from both the application and the system environment.
%
Because information will need to be processed online, it is important
to enable analysis in situ with the application.
%
Because analysis can result in application feedback, query and control
interfaces are required, again to both the application and the system.
%
There exists no general purpose infrastructure that can be programmed,
configured, and launched with the application to provide the
integrated observation, introspection, and adaptation support
required.
%%%%%

%%%%%
This paper presents the \textit{Scalable Observation System (SOS)} for
integrated application monitoring.
%
The SOS design emphasizes a semantic data model with distributed
information management and structured query and access.
%
A dynamic database architecture is used in SOS to support aggregation
of streaming observations from multiple sources.
%
SOS provides interfaces for sources of information to encode data,
metadata, and semantic context.
%
Interfaces are also provided for in situ analytics to acquire
information and send back results for application actuators.
%
SOS launches with the application, runs along side it, and can acquire
its own resources for scalable data collection and processing.
%
The primary objectives of SOS are flexibility, scalability, and
programmability.
%%%%%

%%%%%%%%%%%%%%%%%%%%%%%%%%%%%%%%%%%%%%%%%%%%%%%%%%%%%%%%%%%%%%%%%%%%%%%%%%%%%
%%%%%%%%%%%%%%%%%%%%%%%%%%%%%%%%%%%%%%%%%%%%%%%%%%%%%%%%%%%%%%%%%%%%%%%%%%%%%%
%%%%%%%%%%%%%%%%%%%%%%%%%%%%%%%%%%%%%%%%%%%%%%%%%%%%%%%%%%%%%%%%%%%%%%%%%%%%%%

%%%%%
We now look at our proposal for this new performance model, the
Scalable Observation System, and a working implementation of it
as a programmable middleware layer: SOSflow.
%%%%%

%\todofileend{020\_intro\_motivation.tex}

