
\section{SOS Architectural Model}
%
%SOS contributes a general solution to on-line in situ
%observation, introspection, feedback, and control of scientific
%workflows.
%
Multi-component complex scientific workflows provide a focus for the
general challenge of distributed online monitoring.
%
%Applying the SOS model to the challenge of online observation of
%these workflows led to the development of the SOSflow runtime software
%system.
%

Information from a wide variety of sources is relevent to the
characterization and optimization of a workflow.
%
\par
%
In order to gather run-time information and operate on it, SOS needs
to be active at the same time and in the same context as the workflow
components.
%
This online operation is capable of collecting data from multiple
sources within its context and efficiently servicing requests
for this data.
%
Information captured is distinct and tagged with metadata to enable
classification and automated reasoning.
%
SOS aggregates necessary information together at run-time to
enable high-level reasoning over the entire monitored workflow.
%


\subsection{Components of the SOS Model}
%
The SOS Model consists of the following components:
%
\begin{itemize}
%
\item \textbf{Information Producers} : SOS APIs for getting
  information from different sources to SOS.
%
\item \textbf{Information Management} : SOS online
  information databases/repositiories.
%
\item \textbf{Introspection} : Online access to the
  information databases.
%
\item \textbf{In Situ Analytics} : Components to perform the online
  analysis of the information.
%
\item \textbf{Feedback System} : SOS APIs for sending feedback
  information to non-SOS entities.
%
\end{itemize}


\subsection{Core Features of SOS}
%
%Information from a wide variety of sources can be relevent to the
%characterization and optimization of a workflow.
%
%\par
%
%In order to gather run-time information and operate on it, SOS needs
%to be active at the same time and in the same context as the workflow
%components.
%
%This on-line operation should be able to collect data from multiple
%sources within its context and efficiently service requests
%for this data.
%
%Because information transport technology and methods have such
%diversity and case-by-case optimality, SOS must be implemented
%independent of the technology used in the scientific workflows it
%complements.
%
%All information that is captured in SOS should be distinct and carry
%with it a set of metadata to enable classification and reasoning over
%it.
%
%SOS should aggregate necessary information together at run-time to
%enable high level reasoning over the entire monitored workflow.
%
%\todo[inline]{
%
%
%\par
%
%These SOS requirements inform the features of SOSflow and a frame the
%general solution to the challenge of on-line in situ observation,
%introspection, feedback, and control.
%
\subsubsection{Online}
%
It is necessary to obtain observations at run time to capture features
of workflows that emerge from the interactions of the workflow as a
whole.
%
Relevant features will emerge given a program's interactions
with its problem set, configuration parameters, and execution platform.
%
\subsubsection{Scalable}
%
Tools and run-time infrastructures need to support the largest scale of
operation that applications are being designed for.
%
SOS is targets running at exascale on the next generation of HPC
hardware.
%
SOS is a distributed runtime platform, with an agent present on each
node, using a small fraction of the node's resources.
%
Observation and introspection work is distributed across
the observed application's resources proportionally, and required
performance data aggegation can run concurrently with the job.
%
Node-level SOS agents transfer information off-node using the
high-performance communication infrastructure of the host cluster.
%
Data aggregation need not target a single bottleneck: SOS can support
a scalable number and topology of physical aggregation points in order
to provide timely run-time query access to the global information
space.
%
\subsubsection{Global Information Space}
       Information gathered from applications, tools, and the
       operating system are captured and stored into a common context,
       both on-node and across the entire allocation of nodes.
%
     \begin{itemize}
        %
        \item \textbf{Multiple Perspectives} - The different
          perspectives into the performance space of the workflow can
          be queried to include parts of multiple
          perspectives, helping to contextualize what is seen from one
          perspective with what was happening in another.
          %
        \item \textbf{Time Alignment} - All values captured in SOSflow
          are time-stamped, so that events which occured in the same
          chonological sequence in different parts of the system can be
          aligned and correlated.
          %
        \item \textbf{Reusable Collection} - Information gathered into
          SOSflow can be used for multiple purposes and be correlated
          in various ways without having to be gathered multiple
          times.
          %
        \item \textbf{Unilateral Publish} - Sources of information
          need not coordinate with other workflow or SOSflow
          components about what to publish, they can submit
          information and rely on the SOSflow runtime to decide
          how best to utilize it.
          %
          The SOSflow framework will automatically migrate
          information where it is needed for analysis while managing
          the retention of unused information efficiently.
          %
     \end{itemize}


%\subsection{Locating Features} %----------------------------------------------%
%TODO ---- MORE
%\subsubsection{Application External} %----------------------------------------%
%TODO ---- MORE
%\subsubsection{Asynchronous Communication} %----------------------------------%
%TODO ---- MORE
%\subsubsection{Work Location} %-----------------------------------------------%
%TODO ---- MORE



%%%
%%%  EOF
%%%
