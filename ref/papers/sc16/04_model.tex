
\section{SOSflow Architectural Model}
%
SOS contributes a general solution to on-line in situ
observation, introspection, feedback, and control of scientific
workflows.
%
Multi-component scientific workflows provide a focus for the
general challnge of distributed on-line monitoring.
%
Applying the SOS model to the challenge of on-line observation for
complex and highly variable deployments of scientific workflows led to
the creation of the SOSflow runtime software system.
%
\subsection{Core Features}
%
Information from a wide variety of sources can be relevent to the
characterization and optimization of a workflow.
%
\par
%
In order to gather run-time information and operate on it, SOS needs
to be active at the same time and in the same context as the workflow
components.
%
This on-line operation should be able to collect data from multiple
sources within its context and efficiently service requests
for this data.
%
Because information transport technology and methods have such
diversity and case-by-case optimality, SOS must be implemented
independent of the technology used in the scientific workflows it
complements.
%
All information that is captured in SOS should be distinct and carry
with it a set of metadata to enable classification and reasoning over
it.
%
SOS should aggregate necessary information together at run-time to
enable high level reasoning over the entire monitored workflow.
%
\par
%
These SOS requirements inform the features of SOSflow and a frame the
general solution to the challenge of on-line in situ observation,
introspection, feedback, and control.
%
\subsubsection{Online}
%
It is necessary to obtain observations at run time to capture features
of workflows that emerge from the interactions of the workflow as a
whole.
%
Relevant features will emerge from a program's interactions
with its problem set, its configuration parameters, and with the platform on
which it is executing.

%
\subsubsection{Scalable}
%

%
\subsubsection{Global Information Space}
     \begin{itemize}
        \item \textbf{Multiple Perspective} - ...
        \item \textbf{Time Alignment} - ...
        \item \textbf{Resuable Collection / Unilateral Publish / ???} - ...
     \end{itemize}
\subsection{Locating Features}
\subsubsection{Application External}
\subsubsection{Asynchronous Communication}
\subsubsection{Work Location}

%%%
%%%  EOF
%%%
