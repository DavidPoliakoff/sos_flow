%%%%%%%%%%%%%%%%%%%%%%%%%%%%%%%%%%%%%%%%%%%%%%%%%%%%%%%%%%%%%%%%%%%%%%%%%%%%%%
%%%%%%%%%%%%%%%%%%%%%%%%%%%%%%%%%%%%%%%%%%%%%%%%%%%%%%%%%%%%%%%%%%%%%%%%%%%%%%
%%%%%%%%%%%%%%%%%%%%%%%%%%%%%%%%%%%%%%%%%%%%%%%%%%%%%%%%%%%%%%%%%%%%%%%%%%%%%%
% NOTE: no \IEEEPARstart

%%%%%
\section{Introduction}
Modern clusters for parallel computing are complex environments and
the high-performance applications that run on them do so often with
little insight about their or the system's behavior.
%
This is not to say that information is unavailable.  After all,
sophisticated parallel measurement systems can capture performance and
power data for characterization, analysis, and tuning purposes, but
the infrastructure for observation of these systems is not intended
for general use.
%
Rather, it is specialized for certain types of performance information
and typically does not allow online processing.
%
Other information sources of interest might include the
operating system (OS), network hardware, runtime services, or the
parallel application itself.
%
Cluster monitoring systems like Ganglia \cite{massie2004ganglia} or
Nagios \cite{katsaros2011building} collect and process data about the
performance and health of cluster-wide resources, but do not provide
sufficient fidelity to capture the complex interplay between
applications competing for shared resources.
%
In contrast, the Lightweight Distributed Metric Service
\cite{agelastos2014lightweight} (LDMS) attempts to capture system data
continuously to obtain insight into behavioral characteristics of
individual applications with respect to their resource utilization.
%
However, neither of these provides a framework that can be configured
with and used directly by the application, nor allow for semantic
encoding of multiple observation sources.
%%%%%%%


