
%%%%%%%%%%%%%%%%%%%%%%%%%%%%%%%%%%%%%%%%%%%%%%%%%%%%%%%%%%%%%%%%%%%%%%%%%%%%%%%
%%%%%%%%%%%%%%%%%%%%%%%%%%%%%%%%%%%%%%%%%%%%%%%%%%%%%%%%%%%%%%%%%%%%%%%%%%%%%%%
%%%%%%%%%%%%%%%%%%%%%%%%%%%%%%%%%%%%%%%%%%%%%%%%%%%%%%%%%%%%%%%%%%%%%%%%%%%%%%%
\section{Related Work}
%
Traditionally, HPC research into enhancing performance has been
focused on the low-level efficiency of one application, library, or
a particular machine.
%
%
\par
%
Tools like TAU \cite{huck2006taug} are able to bring HPC developers a
closer look into to their codes and hardware, gathering low-level
performance data and aggregating it for integrated analysis after an
application concludes.
%
Low-level metrics can help identify performance bottlenecks, and are
naturally suited for non-production or offline episodic performance
analysis of individual workflow components.
%
Such deep instrumentation is necessarily invasive and can dictate
rather than capture the observed performance of the instrumented
application when the application is running at scale or required to
engage in significant amounts of interactivity.
%
SOS provides a model that can accept low-level information such as
what TAU collects, while also operating over light-weight higher-level
information suitable for online operation during production runs.
%
\par
%
%\todo[inline]{ Too long to differentiate from another system in related work --- 
%One of our goals, online introspection of workflow performance,
%involves unsupervised machine-learning for model development, and the
%application of complex classifiers to match observations with models.
%%
%Performing principle components analysis on \textit{complex unconditioned
%  performance data} is computationally expensive and unsuitable for a
%runtime environment.
%%
%SOS conditions performance data by annotating it with context and
%semantic tags to help efficiently process it during online
%introspection.}
%
%\par
%
Focused on the needs of large scale data centers, Monalytics
\cite{kutare2010monalytics} demonstrated the utility of
combining monitoring and analytics to rapidly detect and respond to
complex events.
%
SOS takes a similar approach but adopts a general purpose data model,
runtime adaptivity, application configurability, and support for the
integration of heterogenous components for such purposes as analytics
or visualization.
%
\par
%
Falcon \cite{gu1995falcon} proposed a model for online monitoring
and steering of large-scale parallel programs.
%
Where Falcon depended on an application-specific monitoring system
that was tightly integrated with application steering logic and data
visualizations, SOS proposes a loosely-coupled infrastructure that
does not limit the nature or purpose of the information it processes.
%
\par
%
WOWMON \cite{zhang2016wowmon} presented a solution for online
monitoring and analytics of scientific workflows, but imposed several
limitations and lacked generality, particularly with respect to how it
interfaced with workflow components, types of data it could collect
and use, and its server for data management and analytics.
%
\par
%
Online distributed monitoring and aggregation of information is
provided by the DIMVHCM \cite{tesser2012dimvhcm} model, but it
principally services performance understanding through visualization
tools rather than the holistic workflow applications and runtime
environment.
%
DIMVHCM provides only limited support for in situ query of information.
%
\par
%
Cluster monitoring systems like Ganglia \cite{massie2004ganglia} or
Nagios \cite{katsaros2011building} collect and process data about the
performance and health of cluster-wide resources, but do not provide
sufficient fidelity to capture the complex interplay between
applications competing for shared resources.
%
In contrast, the Lightweight Distributed Metric Service
\cite{agelastos2014lightweight} (LDMS) captures system data
continuously to obtain insight into behavioral characteristics of
individual applications with respect to their resource utilization.
%
%However, neither of these provides a framework that can be configured
%with and used directly by the application, nor allow for semantic
%encoding of multiple observation sources.
%
However, neither of these frameworks can be configured with and used
directly by an application.
%
Additionally, they do not allow for richly-annotated information to be
placed into the system from multiple concurrent data sources per node.
%
\par
%
LDMS uses a pull-based interaction model, where a daemon running on
nodes will observe and store a set of values at a regular interval.
%
SOS has a hybrid push-pull model that puts users in control of the
frequency and amount of information exchanged with the runtime.
%
Further, LDMS is currently limited to working with double-precision
floating point values, while SOS allows for the collection of many
kinds of information including JSON objects and ''binary large
object'' (BLOB) data.
%
\par
%
TACC Stats \cite{evans2014comprehensive} facilitates high-level
datacenter-wide logging, historical tracking, and exploration of
execution statistics for applications.  It offers only minimal
runtime interactivity and programmability.
%
\par
%
The related work mentioned here, and many other performance monitoring
tools, are well-implemented, tested, maintained, and regularly used
in production and for performance research studies.
%
However, each have deficiencies that render them unsuitable for a
scalable, general-purpose, online performance analysis framework.
%


%%%
%%%  EOF
%%%
