%%%%%%%%%%%%%%%%%%%%%%%%%%%%%%%%%%%%%%%%%%%%%%%%%%%%%%%%%%%%%%%%%%%%%%%%%%%%%%%
%%%%%%%%%%%%%%%%%%%%%%%%%%%%%%%%%%%%%%%%%%%%%%%%%%%%%%%%%%%%%%%%%%%%%%%%%%%%%%%
%%%%%%%%%%%%%%%%%%%%%%%%%%%%%%%%%%%%%%%%%%%%%%%%%%%%%%%%%%%%%%%%%%%%%%%%%%%%%%%

\title{A Scalable Observation System for Introspection and In Situ Analytics}

\author{
        \IEEEauthorblockN{
                Chad Wood\IEEEauthorrefmark{1},
                Sudhanshu Sane\IEEEauthorrefmark{1},
                Daniel Ellsworth\IEEEauthorrefmark{1},
                Alfredo Gimenez\IEEEauthorrefmark{2},
                Kevin Huck\IEEEauthorrefmark{1},
                Todd Gamblin\IEEEauthorrefmark{2},
                Allen Malony\IEEEauthorrefmark{1}
        }
        \\
        \IEEEauthorblockA{\IEEEauthorrefmark{1}
                          Department of Computer and Information Science\\
                          University of Oregon\\
                          Eugene, OR United States\\
                          Email: \{cdw,ssane,dellswor,khuck,malony\}@cs.uoregon.edu
        }
        \\
        \IEEEauthorblockA{\IEEEauthorrefmark{2}
                          Lawrence Livermore National Laboratory\\
                          Livermore, CA United States\\
                          Email: \{gimenez1,gamblin2\}@llnl.gov
        }
}

\maketitle


%%%%%%%%%%%%%%%%%%%%%%%%%%%%%%%%%%%%%%%%%%%%%%%%%%%%%%%%%%%%%%%%%%%%%%%%%%%%%%%
%%%%%%%%%%%%%%%%%%%%%%%%%%%%%%%%%%%%%%%%%%%%%%%%%%%%%%%%%%%%%%%%%%%%%%%%%%%%%%%
%%%%%%%%%%%%%%%%%%%%%%%%%%%%%%%%%%%%%%%%%%%%%%%%%%%%%%%%%%%%%%%%%%%%%%%%%%%%%%%

\begin{abstract} %------------------------------------------------------------%
SOS is a new model for the online in situ characterization and
analysis of complex high-performance computing applications.
%
SOS employs a data framework with distributed information management
and structured query and access capabilities.
%
The primary design objectives of SOS are flexibility, scalability, and
programmability.
%
SOS provides a complete framework that can be configured with and
used directly by an application, allowing for a detailed workflow
analysis of scientific applications.
%
This paper describes the model of SOS and the experiments used to
validate and explore the performance characteristics of its
implementation in SOSflow.
%
Experimental results demonstrate that SOS is capable of observation,
introspection, feedback and control of complex high-performance
applications, and that it has desirable scaling properties.
%
%


\end{abstract}


\begin{IEEEkeywords} %--------------------------------------------------------%
hpc;
exascale;
in situ;
performance;
monitoring;
introspection;
monalytics;
scientific workflow;
sos;
sosflow;
\end{IEEEkeywords}


\IEEEpeerreviewmaketitle


%%%
%%%  EOF
%%%
